\documentclass[a4paper,12pt]{report}
\usepackage{algorithmic}
\usepackage[linesnumbered,ruled,vlined]{algorithm2e}
\usepackage[margin=2cm]{geometry}
\usepackage[utf8]{inputenc}
\usepackage{listings} 
\usepackage{graphicx} 
\usepackage{color}
\usepackage{xcolor}
\usepackage{hyperref}
%\usepackage{mdframed}

\newcommand\tab[1][1cm]{\hspace*{#1}}
\graphicspath{ {images/} }
\newcommand{\currentdata}{ 1 February 2017}
\newtheorem{example}{Example}

\begin{document}
\vspace{-5cm}
\begin{center}
Department of Computer Science\\
Technical University of Cluj-Napoca\\
\includegraphics[width=10cm]{fig/footer}
\end{center}
\vspace{1cm}
%\maketitle
\begin{center}
\begin{Large}
 %\textbf{Introduction to Artificial Intelligence}\\
\textbf{Intelligent Systems}\\
\end{Large}
\textit{Laboratory activity 2016-2017}\\
\vspace{3cm}
Adrian Groza, Anca Marginean and Radu Razvan Slavescu\\
Tool: Belief and Decision Networks - Version 5.1.10\\
\vspace{1.5cm}
Name:Brincoveanu Vasile Vlad si Ghiurca Dorin\\
Group:30233\\
Email:gg.vladbrincoveanu@gmail.com\\
\vspace{6cm}
Assoc. Prof. dr. eng. Adrian Groza\\
Adrian.Groza@cs.utcluj.ro\\
\vspace{1cm}
\includegraphics[width=10cm]{fig/footer}
\end{center}

\tableofcontents


%\chapter{Laboratory works}

\chapter{Installing the tool ($W_1$)}

\colorbox{blue!20}{\fbox{\begin{minipage}{1.0\textwidth}                                  
The teaching objectives for this week are:
\begin{enumerate}
 \item To know how to use the tool in our future project.
\item To know from where we can download it, what version, etc. And how to run it.
\end{enumerate}
\end{minipage}}}\\

\colorbox{blue!20}{\fbox{\begin{minipage}{16cm}Steps for installing the tool on windows or mac:
\begin{enumerate}
\item First step is to go on the \href{http://www.scikit-learn.org/}{application site}
\item Then you need to go on the installation part of the site.
\item Under installing the latest release section it can be found the linux commands for installing scikit-learn prerequisites.
\end{enumerate}
\end{minipage}}}\\

\begin{itemize}
\item
	\tab Some notes: you should have python (2.7-eq. or greater, or 3.3 - eq. or greater), NumPy (1.8.2 - eq. or greater), SciPy(0.13.3 - eq.  or greater) installed prior to  running the application, otherwise it won't work.\\

\end{itemize}

\chapter{Running and understanding examples ($W_2$)}
\colorbox{blue!20}{\fbox{\begin{minipage}{1.0\textwidth}                                  
The teaching objectives for this week are:
\begin{enumerate}
 \item To run and understand the some examples.
\item To identify what realistic problems are adequate for our tool. 
\end{enumerate}
\end{minipage}}}
\graphicspath{ {fig/} }
\begin{itemize}
\item{First example}

\tab We can use scikit-learn to recognize images of hand-written digits.

\begin{center}
  	\includegraphics[scale=0.8]{exemplu1}
\end{center}

\begin{lstlisting}[frame=single]
"""
================================
Recognizing hand-written digits
================================

An example showing how the scikit-learn can be used to recognize 
images of hand-written digits.

This example is commented in the
:ref:`tutorial section of the user manual <introduction>`.

"""
print(__doc__)

# Author: Gael Varoquaux <gael dot varoquaux at normalesup dot 
org>
# License: BSD 3 clause

# Standard scientific Python imports
import matplotlib.pyplot as plt

# Import datasets, classifiers and performance metrics
from sklearn import datasets, svm, metrics

# The digits dataset
digits = datasets.load_digits()

# The data that we are interested in is made of 8x8 images of 
#digits, let's  have a look at the first 4 images, stored in the 
#`images` attribute of the dataset.  If we were working from
# image files, we could load them using matplotlib.pyplot.imread.  
#Note that each image must have the same size. For these
# images, we know which digit they represent: it is given in the 
#'target' of the dataset.
images_and_labels = list(zip(digits.images, digits.target))
for index, (image, label) in enumerate(images_and_labels[:4]):
    plt.subplot(2, 4, index + 1)
    plt.axis('off')
    plt.imshow(image,cmap=plt.cm.gray_r,interpolation='nearest')
    plt.title('Training: %i' % label)

# To apply a classifier on this data, we need to flatten the 
#image, to turn the data in a (samples, feature) matrix:
n_samples = len(digits.images)
data = digits.images.reshape((n_samples, -1))

# Create a classifier: a support vector classifier
classifier = svm.SVC(gamma=0.001)

# We learn the digits on the first half of the digits
classifier.fit(data[:n_samples //2],digits.target[:n_samples //2])

# Now predict the value of the digit on the second half:
expected = digits.target[n_samples // 2:]
predicted = classifier.predict(data[n_samples // 2:])

print("Classification report for classifier %s:\n%s\n"
      % (classifier, metrics.classification_report(expected, predicted)))
print("Confusion matrix:\n%s" % metrics.confusion_matrix(expected, predicted))

images_and_predictions = list(zip(digits.images[n_samples // 2:], predicted))
for index, (image, prediction) in enumerate(images_and_predictions[:4]):
    plt.subplot(2, 4, index + 5)
    plt.axis('off')
    plt.imshow(image, cmap=plt.cm.gray_r, interpolation='nearest')
    plt.title('Prediction: %i' % prediction)

plt.show()

\end{lstlisting}

\item{Second example - Linear Regression}

\tab This example uses the only the first feature of the diabetes dataset, in order to illustrate a two-dimensional plot of this regression technique. The straight line can be seen in the plot, showing how linear regression attempts to draw a straight line that will best minimize the residual sum of squares between the observed responses in the dataset, and the responses predicted by the linear approximation.

The coefficients, the residual sum of squares and the variance score are also calculated.\\

\begin{center}
  	\includegraphics[scale=0.8]{exemplu2}
\end{center}

\begin{lstlisting}[frame=single]
print(__doc__)


# Code source: Jaques Grobler
# License: BSD 3 clause


import matplotlib.pyplot as plt
import numpy as np
from sklearn import datasets, linear_model
from sklearn.metrics import mean_squared_error, r2_score

# Load the diabetes dataset
diabetes = datasets.load_diabetes()


# Use only one feature
diabetes_X = diabetes.data[:, np.newaxis, 2]

# Split the data into training/testing sets
diabetes_X_train = diabetes_X[:-20]
diabetes_X_test = diabetes_X[-20:]

# Split the targets into training/testing sets
diabetes_y_train = diabetes.target[:-20]
diabetes_y_test = diabetes.target[-20:]

# Create linear regression object
regr = linear_model.LinearRegression()

# Train the model using the training sets
regr.fit(diabetes_X_train, diabetes_y_train)

# Make predictions using the testing set
diabetes_y_pred = regr.predict(diabetes_X_test)

# The coefficients
print('Coefficients: \n', regr.coef_)
# The mean squared error
print("Mean squared error: %.2f"
      % mean_squared_error(diabetes_y_test, diabetes_y_pred))
# Explained variance score: 1 is perfect prediction
print('Variance score: %.2f' % r2_score(diabetes_y_test, diabetes_y_pred))

# Plot outputs
plt.scatter(diabetes_X_test, diabetes_y_test,  color='black')
plt.plot(diabetes_X_test, diabetes_y_pred, color='blue', linewidth=3)

plt.xticks(())
plt.yticks(())

plt.show()
\end{lstlisting}

\end{itemize}
 
\chapter{Understanding conceptual instrumentation ($W_3$)}
\label{ch:tool}

\colorbox{blue!20}{\fbox{\begin{minipage}{1.0\textwidth}                                  
The teaching objectives for this week are:
\begin{enumerate}
 \item  To understand the algorithm(s) on which your tool relies.
\item To get used with writing algorithms in Latex
\end{enumerate}
\end{minipage}}}\\

\begin{itemize}
\item{Explaining the tool}

\tab The Bayesian Belief and Decision Networks applet is a tool to visually solve Bayesian Nets. It has a robust variable elimination algorithm, and allows users to create their own networks and customize the domains and probabilities. The applet has features that allow the user to inspect probabilities, make observations, and monitor nodes. It also allows the user to manually do variable elimination and to inspect the factors created.\\ \footnote{http://www.aispace.org/bayes/help/general.shtml}

\tab The applet also has features to add no-forgetting arcs. There is an independence quiz mode that tests the user on his or her knowledge of the independence rules of Bayesian Nets.\\


\tab The applet can read an XML representation of a Bayesian Network called the XMLBIF format. The application version of the applet saves networks in this format\\

\tab The Verbose Query Window allows the user to manually execute variable elimination. It has a large canvas area to the right and a control panel on the left.\\

\tab During the "Eliminate Variables" stage, there are two ways in which you can choose variables to eliminate. The 'Auto-Eliminate' button will eliminate all the variables in the order specified by the heuristic which is specified by the drop down menu next to 'Heuristic:'. The available heuristics are 'Max-Cardinality', 'Min-Degree', 'Min-Factor', 'Min-Fill', 'Random', and 'Sequential'. The 'Eliminate Next' button will eliminate a single variable each time you press it. The second way to choose variables to eliminate is by clicking on them directly. When you eliminate a variable, it will be greyed out, and the lists of current and eliminated factors will be updated accordingly. \\\footnote{http://www.aispace.org/bayes/help/general.shtml}


\item Algorithm of eliminating variables is:\\

\begin{center}
  	\includegraphics[scale=0.5]{eliminateVar}
\end{center}

\end{itemize}

\chapter{Project description ($W_4$)}


\colorbox{blue!20}{\fbox{\begin{minipage}{1.0\textwidth}                                  
The teaching objectives for this week are:
\begin{enumerate}
 \item To have a clear description of what you intend to develop.
\item To point to specific resources (datasets, knowledge bases, external tools) 
that support the development of your idea and which minimise the risk of failure.
\item To identify related work (articles) that are relevant or similar to your approach.
\end{enumerate}
\end{minipage}}}\\

\section{Describe in natural language}

\tab We propose to determin some infectious diseases of pacients coming to a clinic and what is the probability of them diying. This represents a strong modality to determine the correct diagnostic ,which offers, besides the knowledge and experience of a doctor, a strong mathematical history cumulated togheter to help them.\\

\tab We will model the network starting from an example wich can be found at the following address \href{https://www.norsys.com/tutorials/netica/secA/tut_A1.html}where is presented an example taken for another diseases. The nodes will be choosen depending on our scope wich is to diagnose the patient illness wich arrives at a clinic or hospital with some affections like fever, vomiting states or muscle pain. We will determine based on this symptoms the probability of the patien to hase some specific desease. The nodes of the graph will be chosen so as to encompass the most relevant causes like if the person has been in contact with sick people, is abusing of certain substances, or has been in some regions where are registered many cases the respective desease. Besides these, will result a set of another symptoms relevant to another deseases wich can occure if the illness is not treated. \\

\tab \textbf{At the top level we have the nodes Vizita Africa with the probability of a person visiting Africa of 0.007, with 56 million tourists every year, Contact cu persoane bolnave, with a probability of 0.2, and  consum alcool/tutun, with 30pc of world population smoking and 30pc consuming alcohol daily. Those are independent nodes. A person visiting Africa can get two deadly deseases: Malaria or Yellow Fever, wich are caused by the bite of two types of mosquitoes. Every year are registered around 216 million of malaria cases, 90pc of them being in Africa, and that gives as the probability of a person getting malaria of 0.15. From the node Contact cu persoane bolnave will result flu or laryngitis wich can be caused by some type of viruses. 5 to 20pc of population is getting flu every year and 3.47 in 1000 suffers from laryngitis. Laryngitis can be also produced by alchohol and tobacco consumption. The main symptoms of malaria are coma, vomiting stats and fever. Fever is also caused by yellow fever wich is where the name comes from. Every year are registered aroung 170.000 case of yellow fever, wich gives the probability of getting yellow fever of 0.00013, raported to the population number. Flu is causing fever, pneumonia, cough or breathing problems. The main symptom of laryngitis is breathing problemes. At the bottom is de death node. A person can die of some of the deseases mentioned above.}\\

\textbf{African turism}\\
\tab One of the diseases we try to diagnosticate in this project is malaria. This represents an ifectious disease, largely spread across tropical and subtropical regions. We tried to get the number of persons, which visits annualy Africa and the number of contaminated persons with this disease. Based on statistics, Africa has a anual turism of 56 millions of people. It results that the probability of viziting Africa is =56 mil/7.5mild = 0.007.\footnote{https://qz.com/1023064/africa-is-welcoming-more-tourists-than-ever-before/}\\

\textbf{Malaria}\\
\tab International travellers could be at risk of malaria infection in 91 countries around the world, mainly in Africa, Asia and the Americas. People infected with malaria often experience fever, chills and flu-like illness at first. Left untreated, the disease can lead to severe complications and, in some cases, death. Malaria symptoms appear after a period of 7 days or longer. Fever occurring in a traveller within 3 months of possible exposure is a medical emergency that should be investigated immediately.\\

\tab Malaria is caused by the Plasmodium parasite and is transmitted by female Anopheles mosquitoes which bite between dusk and dawn. There are 5 different types of parasites that infect humans: P. falciparum, P. vivax, P. ovale, P. malariae, and P. knowlesi. Of these, P. falciparum and P. vivax are the most prevalent, and P. falciparum is the most dangerous, with the highest rates of complications and mortality. This deadly form of malaria is a serious public health concern in most countries in sub-Saharan Africa.\\

\tab WHO estimates that 216 million cases of malaria occurred worldwide in 2016 (uncertainty range: 196–263 million) and about 445 000 people died from the disease (uncertainty range: 402 000–486 000), mostly children under 5 years of age in sub-Saharan Africa. Most of the cases in 2015 were in the WHO African Region (90), followed by the WHO South-East Asia Region (7) and the WHO Eastern Mediterranean Region (2). \footnote{
 http://apps.who.int/iris/bitstream/handle/10665/252038/9789241511711\\-eng.pdf;jsessionid=85E7F9D4B20BB3EF77A00BBE23D7231D?sequence=1  (page XVII).\\
 http://www.who.int/malaria/travellers/en/\\
 -http://www.africatravelresource.com/malaria-in-africa/\\
} 

\tab According to a recent report published by the World Health Organization, malaria was responsible for the deaths of 445,000 people in 2016, with 91 per cent of fatalities occurring in Africa. \footnote{https://www.tripsavvy.com/avoid-malaria-when-traveling-in-africa-1454332}\\  

\tab Of the 216 million malaria cases reported in the same year, 90 per cent occurred in Africa. Statistics like these prove that malaria is one of the continent's most deadly diseases - and as a visitor to Africa, you are also at risk. However, with the right precautions, the chances of contracting malaria can be reduced significantly. If there are 216 million cases of malaria and 90per cent of them are in Africa, then 0.9 * 216 million = 194.4 million cases of malaria in Africa. From this probability we can get the probability of getting malaria if you go in Africa:\\
		\tab \tab \textbf{=> p(M|A) = 194.4 mil / 1.26 bil. = 0.15,} \\
1.26 bil. being the population of Africa.\\

\tab According to statistics recorded over several decades, the likelihood of a person being infected with malaria if he did not visit Africa is 2 per cent.\\

\begin{center}
  	\includegraphics[scale=0.8]{malaria.png}
\end{center}

\textbf{Yellow Fever}\\
\tab Yellow fever is caused by the yellow fever virus, which is carried by mosquitoes. It is endemic in 33 countries in Africa and 11 countries in South America. The yellow fever virus can be transmitted by mosquitoes which feed on infected animals in forests, then pass the infection when the same mosquitoes feed on humans travelling through the forest. The greatest risk of an epidemic occurs when infected humans return to urban areas and are fed on by the domestic vector mosquito Aedus aegypti, which then transmits the virus to other humans.\\

\tab Symptoms of yellow fever include fever, headache, jaundice, muscle pain, nausea, vomiting and fatigue. 
 A small proportion of patients who contract the virus develop severe symptoms and approximately half of those die within 7 to 10 days. \\

\tab The virus is endemic in tropical areas of Africa and Central and South America. Large epidemics of yellow fever occur when infected people introduce the virus into heavily populated areas with high mosquito density and where most people have little or no immunity, due to lack of vaccination. In these conditions, infected mosquitoes of the Aedes aegypti specie transmit the virus from person to person. \\

\tab Yellow fever is prevented by an extremely effective vaccine, which is safe and affordable. A single dose of yellow fever vaccine is sufficient to confer sustained immunity and life-long protection against yellow fever disease and a booster dose of the vaccine is not needed. The vaccine provides effective immunity within 30 days for 99 per cent of persons vaccinated. \footnote{ http://www.who.int/mediacentre/factsheets/fs100/en/}\\

\tab Forty seven countries in Africa (34) and Central and South America (13) are either endemic for, or have regions that are endemic for, yellow fever. A modelling study based on African data sources estimated the burden of yellow fever during 2013 was 84 000–170 000 severe cases and 29 000–60 000 deaths.\\

\tab There are 170.000 cases of yellow fever anually in Africa, and to get the probability of yellow fever occuring, we divide the number of registered cases to the Africa’s population:\\

	\tab \tab \textbf{p(FG/A) = 170.000/1.26 mild = 0.00013}\\

\tab The probability to get yellow fever if you not vizit Africa is very small : 0.00001.\\

\begin{center}
  	\includegraphics[scale=0.8]{yellowfever.png}
\end{center}


\textbf{Contact with sick people}\\
\tab We assume that the probability to get contact with sick people is 20 per cent during daily activities.\\

\textbf{Alcohol and smoking}\\
\tab
•	About a third of the male adult global population smokes.\\
•	Smoking related-diseases kill one in 10 adults globally, or cause four million deaths. By 2030, if current trends continue, smoking will kill one in six people.\\
•	Every eight seconds, someone dies from tobacco use.\\
•	Smoking is on the rise in the developing world but falling in developed nations. Among Americans, smoking rates shrunk by nearly half in three decades (from the mid-1960s to mid-1990s), falling to 23 pc of adults by 1997. In the developing world, tobacco consumption is rising by 3.4pc per year.\\
•	About 15 billion cigarettes are sold daily - or 10 million every minute.\\
•	About 12 times more British people have died from smoking than from World War II.\\
•	Cigarettes cause more than one in five American deaths.\\
•	Among WHO Regions, the Western Pacific Region* - which covers East Asia and the Pacific - has the highest smoking rate, with nearly two-thirds of men smoking.\\
•	About one in three cigarettes are consumed in the Western Pacific Region.\\
•	The tobacco market is controlled by just a few corporations - namely American, British and Japanese multinational conglomerates.\\
\begingroup\makeatletter\def\@currenvir{verbatim}
\verbatim
The proportion of the population who consumed alcohol daily declined between
 2007 (8.1%) and 2010 (7.2%) 1. 
•  A higher proportion of 12-17 year olds abstained from alcohol (61.6%) than 
had consumed it in the last 12 months (38.4%) 1. 
•  The proportion of 12-15 year olds and 16-17 year olds abstaining from alcohol
 increased in 2010 (from 69.9% in 2007 to 77.2% in 2010 and from 24.4% to 31.6%, respectively) 1. 
•  In 2010, 1 in 5 people aged 14 years or older consumed alcohol at a level that 
put them at risk of harm from alcohol-related disease or injury over their lifetime, and this remained 
stable between 2007 (20.3%) and 2010 (20.1%). However, the number of people drinking in risky 
quantities increased from 3.5 million in 2007 to 3.7 million in 2010 1. 
•  About 2 in 5 (39.7%) people aged 14 years or older drank, at least once in
 the last 12 months, in a pattern that placed them at risk of an alcohol-related injury from a single 
drinking occasion; but there was a modest by statistically significant decline in risky drinking over the previous
 12 months from 2007 (41.5%) 1. 
•  Males were far more likely than females to consume alcohol in risky quantities, 
and those aged between 18-29 years were more likely than any other age group to consume 
alcohol in quantities that placed them at risk of an alcohol-related injury, and of alcohol-related 
harm over their lifetime 1. 
•  The proportion of pregnant women abstaining during pregnancy increased in 2010 (from 40
% in 2007 to 52% in 2010).
\end{verbatim}


\textbf{According to world-wide statistics, about 7pc of the population regularly consume alcoholic and 30 pc of them tobacco.}\\

\textbf{Coma}\\
\tab The outcome of a patient can be associated with their best response in the first twenty-four hours after injury. Using the Glasgow Coma Scale (3 to 15, with 3 being a person in a coma with the lowest possible score, and 15 being a normal appearing person) research shows that if the best scale is 3 to 4 after twenty four hours, 87pc of those individuals will either die or remain in a vegetative state and only 7pc  will had a moderate disability or good recovery. In patients with a scale from 5 to 7, 53pc  will die or remain in a vegetative state, while 34pc  will have a moderate disability and/or good recovery. In patients with a Glasgow Coma Scale of 8 to 10, 27pc  will die or remain in a coma, while 68pc  will have a moderate disability and/or good recovery. In patients who have a scale from 11 to 15, only 7pc  will be expected to die or remain in a coma, while 87pc  would expect to have at least a moderate disability and/or good recovery (remembering again that this is not an exact science).\footnote{http://www.braininjury.com/coma.shtml}\\

\textbf{Influenza}
\tab Influenza is an acute viral infection that primarily attacks the upper respiratory tract, including the nose, throat, bronchi and, less frequently, the lungs. The disease occurs worldwide and spreads very quickly in populations, especially in crowded circumstances. In the northern hemisphere, annual influenza epidemics occur during autumn and winter affecting approximately 5-20pc of the population.\\

\tab \textbf{5 to 20pc} -- Percentage of the U.S. population that will get the flu, on average, each year.\\
\tab \textbf{200,000} -- Average number of Americans hospitalized each year because of problems with the illness\\
\tab 3,000 to 49,000 -- Number of people who die each year from flu-related causes in the U.S.\\
\tab 10 billion+ -- Average costs of hospitalizations and outpatient doctor visits related to the flu.\\
\tab 1 to 4 days -- Typical time it takes for symptoms to show up once you've caught the virus. Adults can be contagious from the day before symptoms begin through 5 to 10 days after the illness starts.\footnote{http://www.euro.who.int/en/health-topics/communicable-diseases/influenza/data-and-statistics\\
https://www.webmd.com/cold-and-flu/flu-statistics\\
}\\

\tab According to statistics, inluenza affects 12pc of people who make contact with other sick people. In the other cases, influenza affects approximately 4pc  of the people.\\
\tab \tab p(G|CPB) = 12pc , P(G|-CPB) = 4pc , where G represents influenza and CPB represents contact with sick people.\\


\textbf{Laryngitis}

METHODS: 
We retrospectively identified patients with a diagnosis of CL who were seen among a primary care cohort at an urban academic medical center from 2009 to 2010. The incidence of CL was calculated. Symptoms, first-visit treatment, smoking, and demographics were recorded.
RESULTS: 
Of a population of 40,317 people, 280 received a new diagnosis of CL over a 2-year period, representing a yearly incidence of 3.47 cases per 1,000 people. The subjects consisted of 160 women and 120 men. Race was recorded as black (126), Hispanic (47), white (68), or other (39). The mean age was 52.9 years (range, 20 to 90 years). The initial therapies included proton pump inhibitors (79pc, voice therapy (17pc), nasal steroid (13pc), antihistamine (4pc), amitriptyline (4pc), other (17pc), and none (11pc). The most common symptoms were dysphonia (53pc), pain/soreness (45pc), globus sensation (40pc), cough (33pc), excessive throat clearing (28pc), and dysphagia (32pc). An otolaryngologist saw 93pc of the cases.\\
CONCLUSIONS: 
The yearly CL incidence was 3.47 per 1,000 people. Up to 21pc of the population may develop CL in their lifetime. Most of the patients in this cohort were referred to otolaryngologists, and the majority were treated with proton pump inhibitors. Dysphonia, globus sensation, and pain were the most common symptoms. Population surveys could be used to define undiagnosed disease and the overall prevalence of CL.\\

\section {Querries and diagram}

\tab The diagram of all the nodes and combinations of those is here.\\

\begin{center}
  	\includegraphics[scale=0.5]{diagrama}
\end{center}

\tab The result tab with all the probability tables is shown here + a querry on the result of ours from which we can conclude that there is a change of \textbf{0.0009} to be killed by malaria, or yellow fever or flu..etc. The best contributant to this is flu , because it affects almost all the people , while malaria and yellow fever are scarced and dependes on lot of factors to get them.\\

\begin{center}
  	\includegraphics[scale=0.8]{diagramwithProb}
\end{center}


\chapter{Algorithms ($W_8$)}

\begin{itemize}
\item{Neural network models}

\tab Multi-layer Perceptron (MLP) is a supervised learning algorithm that learns a function f(dot): R to the power m goest to  R to the power o by training on a dataset, where m is the number of dimensions for input and o is the number of dimensions for output. Given a set of features X = {x1, x2, ..., xm} and a target y, it can learn a non-linear function approximator for either classification or regression. It is different from logistic regression, in that between the input and the output layer, there can be one or more non-linear layers, called hidden layers. Figure 1 shows a one hidden layer MLP with scalar output. \\ 

\tab The advantages of Multi-layer Perceptron are:

	- capability to learn non-linear models.

	- capability to learn models in real-time (on-line learning) using partial fit.

The disadvantages of Multi-layer Perceptron (MLP) include:

	- MLP with hidden layers have a non-convex loss function where there exists more than one local minimum. Therefore different random weight initializations can lead to different validation accuracy.

	- MLP requires tuning a number of hyperparameters such as the number of hidden neurons, layers, and iterations.

	- MLP is sensitive to feature scaling.  \\

\item{Classifier}

\tab Class MLPClassifier implements a multi-layer perceptron (MLP) algorithm that trains using Backpropagation.

MLP trains on two arrays: array X of size (n samples, n features), which holds the training samples represented as floating point feature vectors; and array y of size (n samples,), which holds the target values (class labels) for the training samples. \\

\tab  Currently, MLPClassifier supports only the Cross-Entropy loss function, which allows probability estimates by running the predict proba method.

MLP trains using Backpropagation. More precisely, it trains using some form of gradient descent and the gradients are calculated using Backpropagation. For classification, it minimizes the Cross-Entropy loss function, giving a vector of probability estimates P(y|x) per sample x. \\

\tab  MLPClassifier supports multi-class classification by applying Softmax as the output function.

Further, the model supports multi-label classification in which a sample can belong to more than one class. For each class, the raw output passes through the logistic function. Values larger or equal to 0.5 are rounded to 1, otherwise to 0. For a predicted output of a sample, the indices where the value is 1 represents the assigned classes of that sample. \\

\end{itemize}

\chapter{Implementation details ($W_5$)}

The teaching objectives for this week are:
\begin{enumerate}
 \item Illustrate each aspect of the reality that you have 
 modelled in your solution.
\item To explain the relevant code from your scenario.
\end{enumerate}

\section{Start example}
\tab We will model the network by startting from an example which can be found \href{https://www.norsys.com/tutorials/netica/secA/tut_A1.html}{here} where is described a case for some diseases.\\

\section{How we chose nodes}
\tab Nodes will be chosen by the chosen scope. We will put a diagnosis of a person who walks to a clinic and presents fever, vomiting, muscle pain, eec. We will try to determine the disease of the pacient by these. Nodes 
will be chosen such that it will have a bigger spectre of posibilities.\\

\section{Probabilities}

\tab To calculate the probability of having fever, the logical "noisy" relationships are used. In propositional logic we can say that fever is true if and only if malaria, yellow fever or flu are true. The noisy-OR model allows for uncertainties as to the ability of each parent to cause his descendant to be true, so a person may have flu but without fever.\\
\tab The model assumes that anything that inhibits a parent to produce a child's effect is independent of any inhibition by other parents to produce offspring effects: for example, any inhibit malaria to cause fever is independent of any flu inhibiting to produce fever. With these assumptions, fever has the false value if and only if all of his parents are inhibited to produce fever, and the probability is the product of inhibition probabilities q for each parent. We assume the following probability of inhibition:\\
\begingroup\makeatletter\def\@currenvir{verbatim}
\verbatim
q_gripa = p(-febra|gripa, -malarie, -febra g) = 0.6;
q_febra g = p(-febra |-gripa, malarie, febra g) = 0.1;
q_malarie  = p(-febra |gripa, malarie, -febra g) = 0.1;
Pe baza acestor informatii si ale asumptiilor noisy-OR,
 se poate construi tabelul de probabilitati de mai sus. Regula generala este:
	P(x_i | parents(X_i)) = 1 – product (given that j:X_j = true)[q_j], 
unde produsul este aplicat parintilor care au valoarea true pentru randul respectiv.

p(febra|malarie) = 1 – q_malarie = 1 – 0.15 = 0.85  
p(febra|febra galbena) = 1 – q_febraGalbena = 1 – 0.1 = 0.9
p(febra|malarie, febra galbena) = 1   - q_malarie,febraGalbena =
 true = 1 – (0.15 * 0.1) = 0.985
p(febra|gripa) = 1 – q_gripa = 1-0.6 = 0.4
p(febra|gripa, febra galbena) = 1 – q_gripa,febraGalbena =
 true = 1 – (0.6 * 0.1) = 0.94
p(febra|gripa, malarie) = 1 – q_gripa,malarie = true   =
 1 – (0.6 * 0.15) = 0.91
p(febra|gripa,malarie,febra galbena) = 1 – q_gripa,malarie,
febra galbena = true = 1 – (0.6*0.1*0.15) = 0.991.

In cazul laringitei, valorile sunt urmatoarele:
q_contact_persoane_bolnave = p(-laringita |
 contact persoane bolnave, -consum tutun/alcool) = 0.22.
q_consum_tutun/alcool = p(-laringita | -contact cu persoane bolnave,
 consum tutun/alcool) = 0.6.

p(laringita | -contact cu persoane bolnave, -consum_tutun/alcool) = 0
p(laringita | consum tutun/alcool) = 1 – q_consum_tutun/alcool=true    = 
 1 – 0.6 = 0.4
p(laringita | contact cu persoane bolnave = 1 – q_contact_persoane_bolnave =
 true   =   1 – 0.22 = 0.78
p(laringita | contact cu persoane bolnave, contact cu persoane bolnave) =
 1 – (0.66 * 0.22) = 0.868.

In cazul starilor de voma, valorile sunt urmatoarele:

q_malarie = p(-stari de voma | malarie, -febra galbena) = 0.34
q_febraGalbena = p(-stari de voma | -malarie, febra galbena) = 0.43

p(stari de voma | -malarie, -febra galbena) = 0
p(stari de voma | -malarie, febra galbena) = 1 – q_febraGalbena = true   =

  1 – 0.43 = 0.57
p(stari de voma | malarie, -febra galbena) = 1 – q_malarie=true   = 1 – 0.34 = 0.66
p(stari de voma | malarie, febra galbena)  = 1 – q_malarie, febraGalbena=true  = 
1  - (0.34 * 0.43) = 0.8538


 In cazul pneumoniei, valorile sunt urmatoarele:
q_gripa = p(-pneumonie | gripa) = 0.28
p(pneumonie | -gripa) = 0.1
p(pneumonie | gripa) = 1  - q_gripa=true   = 1 – 0.28  = 0.72


In cazul tuselor/problemelor respiratorii, valorile sunt urmatoarele:
q_gripa = p(-probleme respiratorii | gripa, -laringita)  = 0.57
q_laringita = p(-probleme respiratorii | -gripa, laringita) = 0.04

p(probleme respiratorii| - laringita, -gripa) = 0
p(probleme respiratorii | -gripa, laringita) = 1 – q_laringita=true  =
 1 – 0.04 = 0.96
p(probleme respiratorii | gripa, -laringita) = 1  - q_gripa=true =
 1 -  0.57 = 0.43
p(probleme respiratorii | gripa, laringita) =1 – q_gripa,laringita=true  = 
1 – (0.57 * 0.04) = 0.9772

In cazul comei, valorile sunt urmatoarele:
q_malarie = p(-coma|malarie) = 0.91
p(coma|malarie) = 1 – q_malarie=true   = 1 – 0.91 = 0.09
p(coma|-malarie) = 0


In case of death, probabilities are the following:
q_febraGalbena = q(-moarte | febraGalbena, -malarie, -laringita, -gripa)  = 0.999992
q_malarie = q(-moarte | -febraGalbena, malarie, -laringita, -gripa)  = 0.9712
q_laringita = q(-moarte | -febraGalbena, -malarie, laringita, -gripa)  = 0.9999993
q_gripa = q(-moarte | -febraGalbena, -malarie, -laringita, gripa)  = 0.999992

p(moarte | -febraGalbena, -malarie, -laringita, -gripa)  = 0
p(moarte | febraGalbena, malarie, laringita, gripa)  = 1 -q_all=true = 
1 – (0.999992 * 0.9712 * 0.9999993 * 0.999992) = 0.0288
p(moarte | febraGalbena, malarie, laringita)  = 1 -q_fg,malarie, laringita=true   = 
 1 – (0.999992 * 0.9712 * 0.9999993) = 0.0288
p(moarte | febraGalbena, malarie, gripa)  = 1 -q_fg,m,g=true  =
 1 – (0.999992 * 0.9712 * 0.999992) = 0.028815
p(moarte | febraGalbena, malarie)  = 1 -q_fg,m = 1 – (0.999992 * 0.9712) = 0.028807
p(moarte | febraGalbena, laringita, gripa)  = 1 – q_fg,l,g=true   = 
  1 – (0.999992 * 0.9999993 * 0.999992) = 0.000008
p(moarte | febraGalbena, laringita)  = 1 – q_fg,l=true   =  
 1 – (0.999992 * 0.9999993) = 0.000008
p(moarte | febraGalbena)  = 1 – q_fg=true   =   1 – 0.999992 = 0.000008
p(moarte | malarie, laringita, gripa)  = 1 -q_m,l,g=true =
 1 – (0.9712 * 0.9999993 * 0.999992) = 0.02880
p(moarte | malarie, laringita)  = 1 -q_m,l,g=true = 1 – (0.9712 * 0.9999993) = 0.0288
p(moarte | malarie, gripa)  = 1 -q_m,g=true = 1 – (0.9712 * 0.9999992) = 0.0288
p(moarte | malarie)  = 1 -q_m=true = 1 – (0.9712) = 0.0288
p(moarte | laringita, gripa)  = 1 -q_l,g=true = 
1 – (0.9999993 * 0.9999992) = 0.000008
p(moarte | laringita)  = 1 -q_l,g=true = 1 – (0.9999993) = 0.000008


\end{verbatim}

\chapter{Experiments and tables ($W_{6}$)}

The objectives for this week are:
\begin{enumerate}
 \item To describe and interpret each experiment that you have performed
\end{enumerate}

An experiment investigates how some variables are related. 
Usually, experiments verify a previosly formulated hypothesis.
Such hypothesis may investigate how your software degrades its performance with larger inputs.
You will need to run simulations to see how your implementation is affected by different inputs.

\section{Result interpretation and data}

\tab Here we will put all the results regarding hypertension and heart-disease from our dataset. We tested it with neural networks, with different algs and with lienarSCV and SCV ot see the diferences. \\
\tab The dataset percentage refeers to the amount of true test data in the dataset. If we let all the rows of the dataset, we will see that every classifier will predict false negative, and false positive, because there is few people with hypertension.\\
\tab We will manipulate the dataset in our favor, to be more precise in thesting the algorithms.\\


\tab Precision (also called positive predictive value) is the fraction of relevant instances among the retrieved instances, while recall (also known as sensitivity) is the fraction of relevant instances that have been retrieved over the total amount of relevant instances.Wiki.\\

\tab The F1 score is the harmonic average of the precision and recall, where an F1 score reaches its best value at 1 (perfect precision and recall) and worst at 0.\\

\tab ‘lbfgs’ is an optimizer in the family of quasi-Newton methods.\\
\tab ‘sgd’ refers to stochastic gradient descent.\\
\tab adam’ refers to a stochastic gradient-based optimizer proposed by Kingma, Diederik, and Jimmy Ba\\


\tab \tab \textbf{Confusion matrix}
\begin{center}
\includegraphics[scale=0.65]{confmatrix.png}
\end{center}

\begin{table}[h!]
\begin{center}
\caption{Results for predicting hypertension}
\label{tab:table1}
\begin{tabular}{c c c c c}
\textbf{Method and algorithm} & \textbf{Dataset percentage(true)} & \textbf{Precision }& \textbf{Recall }& \textbf{F1-score }\\
\hline
Neural Net - lbfgs & 45.6 & 0.69 & 0.69 & 0.69\\
Neural Net - lbfgs & 54.11 & 0.58 & 0.54 & 0.39\\
Neural Net - sgd & 45.6 & 0.53 & 0.52 & 0.38\\
Neural Net - sgd & 54.11 & 0.50 & 0.54 & 0.39\\
Neural Net - adam & 45.6 & 0.57 & 0.53 & 0.37 \\
Neural Net - adam & 54.11 & 0.58 & 0.50 & 0.42 \\
All data NN-adam & 0.09 & 0.82 & 0.89 & 0.83 \\
\end{tabular}
\end{center}
\end{table}

\begingroup\makeatletter\def\@currenvir{verbatim}
\verbatim
 Where confusion matrix of 1.\\
\[414, 157\],\\
\[183, 338\]\\

\tab confusion matrix of 2.\\
\[ 7, 395\],\\
\[ 4, 461\]\\

\tab confusion of 3.
\[558, 13\],\\
\[506, 15\]\\

\tab confusion of 4.
\[ 6, 396\],\\
\[ 7, 458\]\\

\tab conf of 5.
\[566, 5\],\\
\[513, 8\]\\

\tab conf of 6.
\[368, 34\]\\
\[397, 68\]\\
\end{verbatim}


\tab Here we see that there is a kind of good prediction , regarding the small sample of the dataset. Ofc, if we were to put all 50k rows in we will get 95 per cent prediction but only because the neural network have been trained to predict all values false because of dataset having few 5-10 per cent values of true in the test data.\\


\begin{table}[h!]
\begin{center}
\caption{Results for predicting heart-disease}
\label{tab:table2}
\begin{tabular}{c c c c c} 
\textbf{Method and algorithm} & \textbf{Dataset percentage(true)} & \textbf{Precision }& \textbf{Recall }& \textbf{F1-score }\\
\hline
Neural Net - lbfgs & 45.6 & 0.41 & 0.64 & 0.50 \\
Neural Net - lbfgs & 54.11 & 0.53 & 0.73 & 0.61 \\
Neural Net - sgd & 45.6 & 0.63 & 0.64 & 0.51\\
Neural Net - sgd & 54.11 & 0.62 & 0.73 & 0.61\\
Neural Net - adam & 45.6 & 0.66 & 0.44 & 0.37 \\
Neural Net - adam & 54.11 & 0.67 & 0.73 & 0.64 \\
All data NN-adam & 0.09 & 0.91 & 0.95 & 0.93 \\
\end{tabular}
\end{center}
\end{table}


\begingroup\makeatletter\def\@currenvir{verbatim}
\verbatim
Where confusion matrix of 1.\\
\[411, 0\],\\
\[233, 0\]\\

\tab confusion matrix of 2.\\
\[632, 0\],\\
\[237, 0\]\\

\tab confusion of 3.
\[409, 2\],\\
\[230, 3\]\\

\tab confusion of 4.
\[630, 2\],\\
\[236, 1\]\\

\tab conf of 5.
\[ 67, 344\],\\
\[ 16, 217\]\\

\tab conf of 6.
\[617, 15\],\\
\[223, 14\]\\

\tab conf of 7.
\[3554, 1\],\\
\[ 172, 0\]\\

\end{verbatim}



\section{Conclusions}

\tab The default solver ‘adam’ works pretty well on relatively large datasets (with thousands of training samples or more) in terms of both training time and validation score. For small datasets, however, ‘lbfgs’ can converge faster and perform better.Skit.\\

\tab Here we see that there is a kind of good prediction , regarding the small sample of the dataset. Ofc, if we were to put all 50k rows in we will get 95 per cent prediction but only because the neural network have been trained to predict all values false positive because of dataset having few 5-10 per cent values of true in the test data.\\

\tab One fact, if we add hidden layers of neurons, the values will start to converge rapidily to 80+ prediction and go into false positive values, like we said in the upper paragraph.\\

\chapter{Related Work}

\begin{itemize}
\item{Approach}

\tab Our approach was to try different classifier and try and compare which one of those works better for the given dataset of heart disease. The classifiers that we have compared are Linear SVM, Non-linear SVM and Stratified K-Mean on the given vector representation of Cleveland dataset. For our experimental purposes, we have divided our experiment into 2 problems. For both problems, we try to run our classifiers for 60/40 and 80/20 splits where we use 60 per cent and 80 per cent for training our classifiers and 40 per cent  and 20 percent for testing their predictions respectively. Cleveland dataset  has 303 instances and 14 attributes. Our first step is to apply dimensionality reduction and for that purpose we have use Principal Component Analysis (PCA) with 5 components. Once the PCA has been applied on the original X value where X is the feature set, our feature set is reduced to X-new which is a vector representation of 303 samples x 5 features.  First, we try out 60/40 split of X-new where 60 per cent of X-new is used to train the SVM classifier and 40 per cent of X-new is used to test. Similarly, next we try 80/20 split of X-new, which is problem 1 of our experiment. \\

\item{Problem 1}

\tab We classify data using a Linear SVM and predict likelihood of disease belonging to a particular class of severity ranging from 1 to 4 i.e. least to most severe with value of C=0.001. Here, C is the penalty that the classifier incurs every time there is a misclassification that takes place so job of the classifier is to incur penalty as minimal as possible while classifying data in order to keep cost of classification at minimum. In order to check if other type of classifiers work better for this dataset than Linear SVM, we use a RBF i.e. non-linear kernel for the SVM classifier and classify data keeping value of C same. Similarly as last part of our problem 1, we use Stratified k-fold cross validation with 5 folds with a RBF kernel and keeping value of C same as for above classifiers in our search to find which classifier works better for this dataset. The results for this have been shown below in Fig 1 below. \\

\item{Problem 2}

\tab For problem-2 of our experiment, we go a step further by predicting absence (zero) or presence (non-zero) of heart disease. This is possible because we group all severity classes (1 to 4) together which mean that a non-zero would indicate presence of heart disease and a zero would indicate absence of heart disease. Problem-2 of the experiment follows same procedure as that of problem-1. First step is dimensionality reduction for which we use PCA with 5 components that picks best 5 components out of 14 attributes. Now what we get is a vector representation as we obtained in problem-1, which basically implies 303 samples x 5 features. For problem-2, we use an 80/20 split where 80 per cent of data is used to train classifier and 20 per cent is used to test. Now, we follow the same procedure as we did for problem-1 we apply 3 classifiers i.e. Linear SVM, Non-Linear SVM with RBF kernel and Stratified k-means cross validation with 5 folds, all for a value of C=0.001. The results are shown in fig.2. \\

\item{Results}

\begin{center}
  	\includegraphics[scale=0.8]{tabel1.png}
\end{center}

\begin{center}
  	\includegraphics[scale=0.8]{tabel2.png}
\end{center}

\begin{center}
  	\includegraphics[scale=0.8]{tabel3.png}
\end{center}

\begin{center}
  	\includegraphics[scale=0.8]{rezultat1.png}
\end{center}

\item{Conclusion}

\tab Based on the results shown above and experiments performed, it is evident that input data plays an important role in prediction along with machine learning techniques. As is seen in the dataset, provided, we have labels from 0 to 4 where the labels of 4 are hardly 13 and when we split the data into train and test, the number become very less which is nothing but noise and can be totally removed from the dataset by using filtering techniques and hence the linear model will be available to predict the outcome much better with absence of noise. Moreover, PCA has again proven that we can get rid of similar feature set and still obtain predictions with great efficiency. Moreover, we have conducted tests using non linear RBF kernel which is a normal first choice and then validating against linear SVC kernel which outperformed RBF in split case. Most importantly, the above experiment not only helped us in predicting the outcome but also gave us valuable insights about the nature of data, which can be used in future to train our classifiers in a much better way. \\

\tab All source file can be founded  \href{https://github.com/diwakar02/Heart-Disease-Prediction-using-Machine-Leaning}{here.}\\

\end{itemize}

\appendix

\chapter{Your original code}
\label{app:code}

\begingroup\makeatletter\def\@currenvir{verbatim}
\verbatim

<?xml version="1.0" encoding="UTF-8"?>
<BIF VERSION="0.3"  xmlns="http://www.cs.ubc.ca/labs/lci/fopi/ve/XMLBIFv0_3"
	xmlns:xsi="http://www.w3.org/2001/XMLSchema-instance"
	xsi:schemaLocation="http://www.cs.ubc.ca/labs/lci/fopi/ve/XMLBIFv0_3
 http://www.cs.ubc.ca/labs/lci/fopi/ve/XMLBIFv0_3/XMLBIFv0_3.xsd">
<NETWORK>
<NAME>Untitled</NAME>
<PROPERTY>detailed = </PROPERTY>
<PROPERTY>short = </PROPERTY>

<VARIABLE TYPE="nature">
	<NAME>Vizita Africa</NAME>
	<OUTCOME>T</OUTCOME>
	<OUTCOME>F</OUTCOME>
	<OBS>T</OBS>
	<PROPERTY>position = (7291.9384765625, 5058.5302734375)</PROPERTY>
</VARIABLE>

<VARIABLE TYPE="nature">
	<NAME>Stari de voma</NAME>
	<OUTCOME>T</OUTCOME>
	<OUTCOME>F</OUTCOME>
	<PROPERTY>position = (7193.0849609375, 5296.3046875)</PROPERTY>
</VARIABLE>

<VARIABLE TYPE="nature">
	<NAME>Contact cu persoane bolnave</NAME>
	<OUTCOME>T</OUTCOME>
	<OUTCOME>F</OUTCOME>
	<PROPERTY>position = (7511.494140625, 5050.837890625)</PROPERTY>
</VARIABLE>

<VARIABLE TYPE="nature">
	<NAME>Consum alcool/tutun</NAME>
	<OUTCOME>T</OUTCOME>
	<OUTCOME>F</OUTCOME>
	<PROPERTY>position = (7766.93310546875, 5048.2373046875)</PROPERTY>
</VARIABLE>

<VARIABLE TYPE="nature">
	<NAME>Malarie</NAME>
	<OUTCOME>T</OUTCOME>
	<OUTCOME>F</OUTCOME>
	<OBS>T</OBS>
	<PROPERTY>position = (7193.0791015625, 5194.01123046875)</PROPERTY>
</VARIABLE>

<VARIABLE TYPE="nature">
	<NAME>Febra galbena</NAME>
	<OUTCOME>T</OUTCOME>
	<OUTCOME>F</OUTCOME>
	<OBS>T</OBS>
	<PROPERTY>position = (7382.18896484375, 5191.4443359375)</PROPERTY>
</VARIABLE>

<VARIABLE TYPE="nature">
	<NAME>Gripa</NAME>
	<OUTCOME>T</OUTCOME>
	<OUTCOME>F</OUTCOME>
	<PROPERTY>position = (7549.552734375, 5189.99951171875)</PROPERTY>
</VARIABLE>

<VARIABLE TYPE="nature">
	<NAME>Laringita</NAME>
	<OUTCOME>T</OUTCOME>
	<OUTCOME>F</OUTCOME>
	<OBS>F</OBS>
	<PROPERTY>position = (7716.47705078125, 
5183.78759765625)</PROPERTY>
</VARIABLE>

<VARIABLE TYPE="nature">
	<NAME>Febra</NAME>
	<OUTCOME>T</OUTCOME>
	<OUTCOME>F</OUTCOME>
	<PROPERTY>position = (7363.90673828125, 
5301.515625)</PROPERTY>
</VARIABLE>

<VARIABLE TYPE="nature">
	<NAME>Coma</NAME>
	<OUTCOME>T</OUTCOME>
	<OUTCOME>F</OUTCOME>
	<PROPERTY>position = (7086.51318359375, 
5296.1953125)</PROPERTY>
</VARIABLE>

<VARIABLE TYPE="nature">
	<NAME>Tuse/Probleme respiratorii</NAME>
	<OUTCOME>T</OUTCOME>
	<OUTCOME>F</OUTCOME>
	<PROPERTY>position = (7756.91552734375
, 5305.68896484375)</PROPERTY>
</VARIABLE>

<VARIABLE TYPE="nature">
	<NAME>Moarte</NAME>
	<OUTCOME>T</OUTCOME>
	<OUTCOME>F</OUTCOME>
	<PROPERTY>position = (7473.3759765625, 
5437.7626953125)</PROPERTY>
</VARIABLE>

<VARIABLE TYPE="nature">
	<NAME>Pneumonie</NAME>
	<OUTCOME>T</OUTCOME>
	<OUTCOME>F</OUTCOME>
	<PROPERTY>position = (7558.50830078125, 
5310.64404296875)</PROPERTY>
</VARIABLE>

<DEFINITION>
	<FOR>Vizita Africa</FOR>
	<TABLE>0.007 0.993</TABLE>
</DEFINITION>

<DEFINITION>
	<FOR>Stari de voma</FOR>
	<GIVEN>Malarie</GIVEN>
	<GIVEN>Febra galbena</GIVEN>
	<TABLE>0.8538 0.1462 0.66 0.34 0.57 0.43 0.0 1.0</TABLE>
</DEFINITION>

<DEFINITION>
	<FOR>Contact cu persoane bolnave</FOR>
	<TABLE>0.2 0.8</TABLE>
</DEFINITION>

<DEFINITION>
	<FOR>Consum alcool/tutun</FOR>
	<TABLE>0.38 0.62</TABLE>
</DEFINITION>

<DEFINITION>
	<FOR>Malarie</FOR>
	<GIVEN>Vizita Africa</GIVEN>
	<TABLE>0.15 0.85 0.002 0.998</TABLE>
</DEFINITION>

<DEFINITION>
	<FOR>Febra galbena</FOR>
	<GIVEN>Vizita Africa</GIVEN>
	<TABLE>1.3E-4 0.99987 1.0E-5 0.99999</TABLE>
</DEFINITION>

<DEFINITION>
	<FOR>Gripa</FOR>
	<GIVEN>Contact cu persoane bolnave</GIVEN>
	<TABLE>0.12 0.88 0.04 0.96</TABLE>
</DEFINITION>

<DEFINITION>
	<FOR>Laringita</FOR>
	<GIVEN>Contact cu persoane bolnave</GIVEN>
	<GIVEN>Consum alcool/tutun</GIVEN>
	<TABLE>0.868 0.132 0.78 0.22 0.4 0.6 0.0 1.0</TABLE>
</DEFINITION>

<DEFINITION>
	<FOR>Febra</FOR>
	<GIVEN>Malarie</GIVEN>
	<GIVEN>Febra galbena</GIVEN>
	<GIVEN>Gripa</GIVEN>
	<TABLE>0.991 0.009 0.985 0.015 0.91 0.09 0.85 0.15 0
.94 0.06 0.9 0.1 0.4 0.6 0.0 1.0</TABLE>
</DEFINITION>

<DEFINITION>
	<FOR>Coma</FOR>
	<GIVEN>Malarie</GIVEN>
	<TABLE>0.09 0.91 0.0 1.0</TABLE>
</DEFINITION>

<DEFINITION>
	<FOR>Tuse/Probleme respiratorii</FOR>
	<GIVEN>Gripa</GIVEN>
	<GIVEN>Laringita</GIVEN>
	<TABLE>0.9772 0.0228 0.43 0.57 0.96 0.04 0.0 1.0</TABLE>
</DEFINITION>

<DEFINITION>
	<FOR>Moarte</FOR>
	<GIVEN>Malarie</GIVEN>
	<GIVEN>Febra galbena</GIVEN>
	<GIVEN>Gripa</GIVEN>
	<GIVEN>Laringita</GIVEN>
	<TABLE>0.0288 0.9712 0.0288 0.9712 0.0288 0.9712 0.0288 0.9712 0.0288 
0.9712 0.0288 0.9712 0.0288 0.9712 0.0288 0.9712 1.0E-5 0.99999 1.0E-5 0.9999
 1.0E-5 0.99999 1.0E-5 0.99999 1.0E-5 0.99999 1.0E-5 0.99999 1.0E-5 0.99999 
0.0 1.0</TABLE>
</DEFINITION>

<DEFINITION>
	<FOR>Pneumonie</FOR>
	<GIVEN>Gripa</GIVEN>
	<TABLE>0.72 0.28 0.1 0.9</TABLE>
</DEFINITION>
</NETWORK>
</BIF>


\end{verbatim}

\end{document}
