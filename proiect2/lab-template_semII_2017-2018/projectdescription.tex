\section{Describe in natural language}

\tab We propose to predict the chance of having a heart disease or having hypertension given the dataset.\\
\tab Our approach was to use neural networks and to try different algorithms integrated in Multi-layer Perceptron (MLP) and compare which one of those works better for the given dataset of heart disease. More detailed info will be described in implementation chapter.\\

\tab We will start by examinin our dataset .\footnote{https://www.kaggle.com/asaumya/healthcare-problem-prediction-stroke-patients} . The datasets are provided by SaumyaAgarwal HealthCare Problem: Prediction Stroke Patients on the kaggle website. He used it to predict stroke by having different data. We will use it for predicting  heart disease or having hypertension.\\

\tab Understanding Data
Here is the Definitions of the columns of the data

id-Patient ID\\
gender-Gender of Patient\\
age-Age of Patient\\
hypertension-0 - no hypertension, 1 - suffering from hypertension\\
heart-disease-0 - no heart disease, 1 - suffering from heart disease\\
ever-married-Yes/No\\
work-type-Type of occupation\\
Residence-type-Area type of residence (Urban/ Rural)\\
avg-glucose-level-Average Glucose level (measured after meal)\\
bmi-Body mass index\\
smoking-status-patient’s smoking status\\
stroke-0 - no stroke, 1 - suffered stroke\\

\tab The columns that we will be using are all except id-patient and stroke, to not interfere with what our poster used the data for.\\
\tab Furthermore, there are 2 datasets, one for training and one for testing. We merged all , and use cross-validation function to split the data into test data and train data.\\

\tab All this data will be converted to integers, usign different functions of the panda library.\\ 