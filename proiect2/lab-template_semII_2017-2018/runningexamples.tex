\graphicspath{ {fig/} }
\begin{itemize}
\item{First example}

\tab We can use scikit-learn to recognize images of hand-written digits.

\begin{center}
  	\includegraphics[scale=0.8]{exemplu1}
\end{center}

\begin{lstlisting}[frame=single]
"""
================================
Recognizing hand-written digits
================================

An example showing how the scikit-learn can be used to recognize 
images of hand-written digits.

This example is commented in the
:ref:`tutorial section of the user manual <introduction>`.

"""
print(__doc__)

# Author: Gael Varoquaux <gael dot varoquaux at normalesup dot 
org>
# License: BSD 3 clause

# Standard scientific Python imports
import matplotlib.pyplot as plt

# Import datasets, classifiers and performance metrics
from sklearn import datasets, svm, metrics

# The digits dataset
digits = datasets.load_digits()

# The data that we are interested in is made of 8x8 images of 
#digits, let's  have a look at the first 4 images, stored in the 
#`images` attribute of the dataset.  If we were working from
# image files, we could load them using matplotlib.pyplot.imread.  
#Note that each image must have the same size. For these
# images, we know which digit they represent: it is given in the 
#'target' of the dataset.
images_and_labels = list(zip(digits.images, digits.target))
for index, (image, label) in enumerate(images_and_labels[:4]):
    plt.subplot(2, 4, index + 1)
    plt.axis('off')
    plt.imshow(image,cmap=plt.cm.gray_r,interpolation='nearest')
    plt.title('Training: %i' % label)

# To apply a classifier on this data, we need to flatten the 
#image, to turn the data in a (samples, feature) matrix:
n_samples = len(digits.images)
data = digits.images.reshape((n_samples, -1))

# Create a classifier: a support vector classifier
classifier = svm.SVC(gamma=0.001)

# We learn the digits on the first half of the digits
classifier.fit(data[:n_samples //2],digits.target[:n_samples //2])

# Now predict the value of the digit on the second half:
expected = digits.target[n_samples // 2:]
predicted = classifier.predict(data[n_samples // 2:])

print("Classification report for classifier %s:\n%s\n"
      % (classifier, metrics.classification_report(expected, predicted)))
print("Confusion matrix:\n%s" % metrics.confusion_matrix(expected, predicted))

images_and_predictions = list(zip(digits.images[n_samples // 2:], predicted))
for index, (image, prediction) in enumerate(images_and_predictions[:4]):
    plt.subplot(2, 4, index + 5)
    plt.axis('off')
    plt.imshow(image, cmap=plt.cm.gray_r, interpolation='nearest')
    plt.title('Prediction: %i' % prediction)

plt.show()

\end{lstlisting}

\item{Second example - Linear Regression}

\tab This example uses the only the first feature of the diabetes dataset, in order to illustrate a two-dimensional plot of this regression technique. The straight line can be seen in the plot, showing how linear regression attempts to draw a straight line that will best minimize the residual sum of squares between the observed responses in the dataset, and the responses predicted by the linear approximation.

The coefficients, the residual sum of squares and the variance score are also calculated.\\

\begin{center}
  	\includegraphics[scale=0.8]{exemplu2}
\end{center}

\begin{lstlisting}[frame=single]
print(__doc__)


# Code source: Jaques Grobler
# License: BSD 3 clause


import matplotlib.pyplot as plt
import numpy as np
from sklearn import datasets, linear_model
from sklearn.metrics import mean_squared_error, r2_score

# Load the diabetes dataset
diabetes = datasets.load_diabetes()


# Use only one feature
diabetes_X = diabetes.data[:, np.newaxis, 2]

# Split the data into training/testing sets
diabetes_X_train = diabetes_X[:-20]
diabetes_X_test = diabetes_X[-20:]

# Split the targets into training/testing sets
diabetes_y_train = diabetes.target[:-20]
diabetes_y_test = diabetes.target[-20:]

# Create linear regression object
regr = linear_model.LinearRegression()

# Train the model using the training sets
regr.fit(diabetes_X_train, diabetes_y_train)

# Make predictions using the testing set
diabetes_y_pred = regr.predict(diabetes_X_test)

# The coefficients
print('Coefficients: \n', regr.coef_)
# The mean squared error
print("Mean squared error: %.2f"
      % mean_squared_error(diabetes_y_test, diabetes_y_pred))
# Explained variance score: 1 is perfect prediction
print('Variance score: %.2f' % r2_score(diabetes_y_test, diabetes_y_pred))

# Plot outputs
plt.scatter(diabetes_X_test, diabetes_y_test,  color='black')
plt.plot(diabetes_X_test, diabetes_y_pred, color='blue', linewidth=3)

plt.xticks(())
plt.yticks(())

plt.show()
\end{lstlisting}

\end{itemize}